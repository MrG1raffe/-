\documentclass[../TV&MS.tex]{subfiles}
\begin{document}
    
\section{Центральная предельная теорема}

С места в карьер:

\begin{Def}
    Последовательность случайных величин $\{ \xi_k \}$ \uline{удовлетворяет ЦПТ}, если $\exists \{ a_k \} \in \Real, \{ b_k \} > 0$, такие что 
    \[
        \dfrac{S_k - a_k}{b_k} \xrightarrow[]{d} \xi \sim \Norm(0, 1)
    .\] 
\end{Def}

Разомнемся на чем-нибудь попроще.

\begin{Th}
    Пусть $\{ \xi_k \}$ "--- норсв с конечным матожиданием и конечной ненулевой дисперсией. Тогда $\{ \xi_k \}$ удовлетворяет ЦПТ, причем $a_n = \Expec S_n$,
    a $b_n = \sqrt{\Disp S_n}$, где  $S_n = \sum\limits_{k=1}^{n} \xi_k$.
\end{Th}

\begin{Note}
    Здесь распределение $\dfrac{S_k - \Expec \xi_k}{\sqrt{\Disp \xi_k}}$ сходится к стандартному нормальному не абы как, а очень даже равномерно по $x$ на всей действительной прямой.
\end{Note} 

\begin{Proof}
    Без ограничения общности будем считать, что $\Expec \xi_k = 0$ и  $\Disp \xi_k = 1$.
    Тогда 
     \[
         f(t) := \Expec e^{it\xi_k} = 1 + \cancelto{\text{\scriptsize{0}}}{ita} \quad - \dfrac{t^2}{2} + \overline{o} \left( \dfrac{1}{n} \right) 
    .\]
    \[
        f_{\frac{S_n}{\sqrt{n}}} (t) = \Expec e^{it\frac{S_n}{\sqrt{n}}} = 
        \left( f \left( \dfrac{t}{\sqrt{n}} \right) \right)^{n} = 
        \left( 1 - \dfrac{t^2}{2n} + \overline{o} \left( \dfrac{1}{n} \right)  \right)^{n} 
        \xrightarrow[n \rightarrow \infty]{} e^{-\frac{t^2}{2}}
    .\] 
    Таким образом, харфункция центрированной нормированной случайной величины сходится слабо к харфункции стандартного нормального распределения, а значит, центрированная нормарованная случайная величина сходится по распределению к $\Norm(0, 1)$.
\end{Proof} 

\begin{Why}
    Если вам уже стало плохо, то переживать не стоит "--- дальше будет еще хуже
\end{Why} 

С этого момента положим:
\begin{itemize}
    \item Случайные величины $\{ \xi_k \}$ независимы и определены на одном вероятностном пространстве
    \item $F_k(x) = \Pro (\xi_k < x)$
    \item $S_n = \sum\limits_{k=1}^{n} \xi_k$
    \item $a_k = \Expec \xi_k$, $A_n = \sum\limits_{k=1}^{n} a_k$ 
    \item $b_k^2 = \Disp \xi_k$, $B_n^2 = \sum\limits_{k=1}^{n} b_k^2$
\end{itemize} 

\begin{Th}[Ляпунова]
    Пусть $\mu_k^3 := \Expec \left| \xi_k - a_k \right|^3 < \infty, \quad
    M_n^3 := \sum\limits_{k=1}^{n} \mu_k^3$. Пусть выполнено \uline{условие Ляпунова}:
    \begin{equation}\label{Lyapunov}
        \dfrac{M_n^3}{B_n^3} \xrightarrow[n \rightarrow \infty]{} 0
    \end{equation}
    Тогда $\{ \xi_k \}$ удовлетворяет ЦПТ.
\end{Th}

\begin{Th}[Линдеберга]
    Пусть $\varepsilon > 0$. Тогда если  $a_k < \infty$, $b_k < \infty$ и выполнено \uline{условие Линдеберга}:
    \begin{equation}\label{Lindeberg}
        L_n(\varepsilon) = \dfrac{1}{B_n^2} \sum\limits_{k=1}^{n} 
        \int\limits_{ \left| x - a_k \right| > \varepsilon B_n}^{}
        \left( x - a_k \right)^2 dF_k(x) \xrightarrow[n \rightarrow \infty]{}
        0,
    \end{equation} 
    То $\{ \xi_k \}$ удовлетворяет ЦПТ.
\end{Th} 

\begin{Note}
    Если случайные величины одинаково распределены, то условие Линдеберга эквивалентно существованию дисперсии.
\end{Note}

\begin{Def}
    $\{ \xi_k \} $ удовлетворяют \uline{условию Феллера}, если
    \begin{equation}\label{Feller}
        \forall \varepsilon > 0 \quad
        \max\limits_{1 \leqslant k \leqslant n}
        \Pro \left( \dfrac{ \left| \xi_k - a_k \right| }{B_n} \right) > \varepsilon
        \xrightarrow[n \rightarrow \infty]{} 0. 
    \end{equation}
\end{Def} 

\begin{Th}
\[
    \begin{cases}
        \text{ЦПТ} \\
        (\ref{Feller})
    \end{cases}
    \iff (\ref{Lindeberg}).
\]
\end{Th} 

\begin{Proof}
    Покажем, что из (\ref{Lindeberg}) следует (\ref{Feller}).
    \[    
        \max_{1 \leqslant k \leqslant n} \Pro \left( \frac{ \left| \xi_k - a_k \right|}{B_n} > \varepsilon \right) \leqslant 
        \max_{1 \leqslant k \leqslant n} \frac{b_k^2}{\varepsilon^2 B_n^2} = 
        \max_{1 \leqslant k \leqslant n} \frac{b_k^2}{\varepsilon^2 \left( b_1^2 +  \ldots + b_n^2 \right) }
    \]

    \begin{multline*}
        \max_{1 \leqslant k \leqslant n} \Pro \left( \left| \xi_k - a_k \right| > \varepsilon B_n \right)  \leqslant
        \sum\limits_{k=1}^{n} \Pro \left( \left| \xi_k - a_k \right| > \varepsilon B_n \right) = 
        \sum\limits_{k=1}^{n} \int\limits_{ \left| x - a_k \right| > \varepsilon B_n}^{} 1 dF_k(x) = \\
        \sum\limits_{k=1}^{n} \int\limits_{ \left| x - a_k \right| > \varepsilon B_n}^{} \frac{(x - a_k)^2}{(x - a_k)^2} dF_k(x) \leqslant
        \frac{1}{\varepsilon^2 B_n^2} \int\limits_{ \left| x - a_k \right| > \varepsilon B_n}^{} (x - a_k)^2 dF_k(x) \xrightarrow[n \rightarrow \infty]{} 0.
    \end{multline*} 
    Таким образом, условие Линдеберга влечет за собой условие Феллера.
\end{Proof} 

\begin{Th}
    Условие Ляпунова (\ref{Lyapunov}) влечет за собой условие Линдеберга (\ref{Lindeberg}).
\end{Th}

\begin{Proof}
    \begin{multline*}
        L_n(\varepsilon) = \frac{1}{B_n^2} \sum\limits_{k=1}^{n}  \int\limits_{ \left| x - a_k \right| > \varepsilon B_n}^{} (x - a_k)^2 dF_k(x) = 
        \frac{1}{B_n^2} \sum\limits_{k=1}^{n} \int\limits_{ \left| x - a_k \right| > \varepsilon B_n}^{} \frac{|x - a_k|^3}{|x - a_k|} dF_k(x) \leqslant\\
        \leqslant \frac{1}{\varepsilon B_n^3} \sum\limits_{k=1}^{n} \int\limits_{-\infty}^{+\infty} \left| x - a_k \right|^3 dF_k(x) = \frac{M_n^3}{\varepsilon B_n^3} \xrightarrow[n \rightarrow \infty]{} 0
    .\end{multline*}
\end{Proof} 

\begin{Th}[Неравенство Берри"--~Эссена]
    $\sup\limits_{x} \left| \Pro \left( \dfrac{S_n - A_n}{B_n} < x \right) - \Phi(x) \right| \leqslant C\dfrac{M_n^3}{B_n^3}$. \\
    Если случайные величины одинаково распределены, то правую часть можно переписать так:
    \[
    C \frac{M_n^3}{B_n^3} = C_0 \frac{n \mu_k^3}{n^{3/2} b_k^3} =
    C_0 \frac{\mu_k^3}{b_k^3 \sqrt{n}}
    .\] 
    Константу $C_0$ уточняли, уточняли, уточняли и в конце концов доуточнялись до $C_0 \geqslant \dfrac{\sqrt{10} + 3}{6 \sqrt{2 \pi}}$. А потом доказали, что здесь вообще стоит строгое равенство.
\end{Th}

\newpage

\end{document}
