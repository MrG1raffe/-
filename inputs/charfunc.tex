\documentclass[../TV&MS.tex]{subfiles}

\begin{document}
\section{Характеристические функции}

Введем еще какие-нибудь понятия, которые что-то там нам облегчат (наверное).

\begin{Def}
    Пусть $\xi$ "---~случайная величина. Тогда будем говорить, что $f_\xi \left( x \right) $ "---~\uline{характеристическая функция $\xi$}, если:
    \[
        f_\xi(t) = \Expec e^{it\xi}
    .\] 
\end{Def} 

Поясним раз.  $\Expec e^{it\xi}$, чтобы не сломать мозг раньше времени, понимаем в смысле
$\Expec e^{it\xi} = \Expec \cos{t\xi} + i\Expec \sin{t\xi}$.

Поясним два. Если $\xi$ "---~дискретная случайная величина, то она задается рядом
распределения $\Pro(\xi = x_k)$, тогда  $f_\xi(t) = \sum\limits_{k}^{} e^{itx_k} \Pro(\xi = x_k)$. Если же $\xi$"---~абсолютно непрерывная случайная величина, 
то  $f_\xi(t) = \int\limits_{}^{} e^{itx} dF_\xi(x)$, где $F_\xi(x)$"---~функция распределения $\xi$, а интеграл как всегда берется по всему пространству, да еще и Лебегов.

Прежде чем мы немного поковыряем свойства харфункции и посмотрим примерчики,
введем еще пару определений, ведь без определений так скучно жить!

\begin{Def}
    Функция $f(x)$ называется \uline{неотрицательно определенной}, если:
    \[
    \forall n \in \Nat \quad \forall t_1, \ldots t_n \in \Real \quad
    \forall z_1, \ldots z_n \in \Compl \quad 
    \sum\limits_{i,j=1}^{n} f(t_i - t_j) z_i \overline{z_j} \geqslant 0
    .\] 
\end{Def} 

\begin{Def}
    Случайная величина $\xi$ имеет \uline{решетчатое распределение}, если
    $\exists a, b \colon \sum\limits_{-\infty}^{+\infty}
    \Pro\left(\xi = a + bk\right)= 1$. Тогда число $b$ называется 
    \uline{шагом распределения}.
\end{Def} 

Свойства (под $f(t)$ подразумевается $f_\xi(t)$ для какой-то случайной величины $\xi$):

\begin{enumerate}
    \item $ \left| f(t) \right| \leqslant 1 $.
    \item $f(0) = 1$.
    \item $f(-t) = \overline{f(t)}$ 
    \item $\forall t \in \Real \quad f_\xi(t) \in \Real \iff$  $\xi$ распределена симметрично (тривиальное следствие из предыдущего).
    \item\label{ravnepr} $f(t)$ равномерно непрерывна на $\Real$.
    \item Если $\eta = a\xi + b$, то $f_\eta(t) = e^{itb}f_\xi(at)$.
    \item\label{nez} Если $\xi_1,  \ldots , \xi_n$"---~независимые случайные величины, и 
    $\eta = \xi_1 +  \ldots + \xi_n$, то $f_\eta(x) = \prod\limits_{k=1}^{n} 
    f_{\xi_k}(x)$.
    \item\label{moments} Если $\Expec \left|\xi^k\right| < \infty$, то 
    $\Expec \xi^k = i^k f^{(k)}_\xi(0)$.
    Если $k$ четно, то верно и обратное утверждение.
    \item \begin{Th}[Бохнера-Хинчина]
            $f(t)$ является характеристической функцией $\iff$ $f(0) = 1$ 
            и  $f(t)$ обладает свойством неотрицательной определенности.
          \end{Th}
    \item $ \left| f_\xi(t) \right|$ интегрируема 
         $\implies p_\xi(x) \xrightarrow[ \left| x \right| \rightarrow \infty]{} 0$,
         где $p_\xi(x)$ "--- плотность распределения.
    \item Случайная величина $\xi$ имеет решетчатое распределение с шагом
        $b \iff \left| f_\xi \left( \dfrac{2\pi}{b} \right) \right| = 1$. 
\end{enumerate}

Не расслабляться! Сейчас докажем некоторые утверждения.

\begin{Proof} \eqref{ravnepr}
\begin{multline*}
    \bigl| f(t + h) - f(t) \bigr| = 
    \left| \int\limits_{-\infty}^{\infty} e^{i(t+h)x}dF(x) + 
    \int\limits_{-\infty}^{\infty} e^{itx}dF(x) \right| \leqslant
    \int\limits_{-\infty}^{\infty} \Bigl| e^{itx} \left( e^{ith} + 1 \right)  \Bigr| dF(x) \leqslant \\
    \leqslant \int\limits_{-\infty}^{\infty} \Bigl| e^{ihx} - 1 \Bigr| dF(x) =
    \underbrace{\int\limits_{|x| \leqslant M}^{} \left| e^{ith} - 1 \right| dF(x)}_{I_1} + 
    \underbrace{\int\limits_{|x| > M}^{} \left| e^{ith}-1 \right|dF(x)}_{I_2} 
.\end{multline*}
Оценим теперь интегралы $I_1$ и $I_2$.
Функция, неепрерывная на компакте равномерно непрерывна на нем, а значит
$\forall \varepsilon > 0 \quad \exists h$, такой что $I_1 < \dfrac{\varepsilon}{2}$.
Для второго интеграла имеем: $ \left| e^{ith} - 1 \right|\leqslant 2 \implies
I_2 \leqslant 2\Pro \left( \left| \xi \right| > M \right) < \dfrac{\varepsilon}{2}$ за счет выбора $M$.
\end{Proof} 

\begin{Proof} \eqref{nez}

Воспользуемся независимостью случайных величин, чтобы разбить одно большое матожидание на много маленьких.
    $$
        f_{\xi_1 +  \ldots + \xi_n} =
        \Expec e^{it(\xi_1 + \ldots + \xi_n)} = 
        \Expec \left( e^{it\xi_1} \ldots e^{it\xi_n} \right) =
        \Expec e^{it\xi_1}  \ldots \Expec e^{it\xi_n} =
        \prod\limits_{k=1}^{n} f_{\xi_k}(x)
    .$$ 
\end{Proof} 

\begin{Proof} \eqref{moments}

    Первую производную посчитаем по определению:
    \[
        f_\xi'(t) = \lim\limits_{h \rightarrow 0} \frac{f_\xi(t + h) - f_\xi(t)}{h}=
        \lim\limits_{h \rightarrow 0} \int\limits_{-\infty}^{\infty} e^{ixt} \frac{e^{ixh} - 1}{h} dF_\xi(x) = \ldots 
    \]
    Тут у нас $\dfrac{e^{ixh} - 1}{h} \leqslant \left| x \right|$,
    поэтому интеграл сходится равномерно (признак Вейерштрасса),
    и мы можем поменять местами предел и интеграл и ничего нам за это не будет:
    \[
        \ldots = \int\limits_{-\infty}^{\infty} e^{ixt} 
        \lim\limits_{h \rightarrow 0} \frac{e^{ixh} - 1}{h} dF_\xi(x) = \ldots
    \] 
    Теперь надо присатльно посмотреть на последнее подынтегральное выражение и заметить там 3-ий замечательный предел:
    \[
        \frac{e^{ixh} - 1}{h} \xrightarrow[h \rightarrow 0]{} ix
    .\] 
    Тогда продолжаем:
    \[
        \ldots = i \int\limits_{-\infty}^{\infty} xe^{itx}dF_\xi(x)
    .\]
    При $t = 0 \  e^{itx} = 1$, поэтому
    \[
        f_\xi'(0) = i \int\limits_{-\infty}^{\infty} xdF(x) = i \Expec \xi
    .\]
    По индукции показываем справедливость для производных старших порядков.
    Обратное утверждение для четных $k$ доказывать не будем. Но там немного полопиталить и доказать это все безобразие по индукции. Можно залезть в Ширяева и удовлетворить свое любопытство.
\end{Proof}

\begin{Why}
    Мяу. Если $\Expec \left| \xi \right|^n < \infty$, то мы можем записать харфункцию в виде суммы:
    \[
        f_\xi(t) = \sum\limits_{k=0}^{n} \frac{t^k}{k!}f_\xi^{(k)}(0) +\overline{o}(t^{k}) =
        1 + \sum\limits_{k=1}^{n} \frac{(it)^k}{k!} \Expec \xi^k + \overline{o}(t^{k}) 
    .\]
    Довольно удобно считать моменты, если мы уверены в их существовании.
\end{Why} 

Теперь рассмотрим некоторые примеры:

\begin{Ex}
    Найдем характеристическую функцию для стандартного нармального распределения $\Norm(0, 1)$
    с плотностью $p(x) = \dfrac{1}{\sqrt{2\pi}} e^{\frac{-x^2}{2}}$.
    \begin{multline*}
        f(t) = \Expec e^{it\xi} = 
        \frac{1}{\sqrt{2\pi}}\int\limits_{-\infty}^{\infty} e^{itx} e^{\frac{-x^2}{2}} dx =
        \frac{1}{\sqrt{2\pi}} \int\limits_{-\infty}^{\infty} 
        e^{-\frac{1}{2}(x - it)^2} e^{-\frac{t^2}{2}}dx = \\
        = \frac{1}{\sqrt{2\pi}} e^{-\frac{t^2}{2}} \int\limits_{-\infty}^{\infty}  e^{-\frac{1}{2}(x - it)^2}d(x - it) =
        e^{-\frac{t^2}{2}}
    .\end{multline*}
\end{Ex}

\begin{Ex}
    Вот представь ситуацию: входишь ты в хату, а тебе пахан кидает под ноги
    кидает $\cos t^2$ и говорит: <<а найди-ка нам случайную величину, для 
    которой это выражение будет харфункцией>>. Тут главное "--- не зашквариться
    и по понятиям пояснить, что если бы $\cos t^2$ была бы харфункцией, то по
    свойству~\eqref{ravnepr} она была бы равномерно непрерывной на $\Real$,
    а это не так: если $t_1^2 = 2\pi k - \dfrac{\pi}{2}$, а $t_2^2 = 2\pi k$,
    то $\cos t_2^2 - \cos t_1^2 = 1$, а
    \[
        % \vphantom чтобы выровнять знаки корня по высоте
        t_2 - t_1 =\sqrt{\vphantom{\frac{\pi}{2}} 2\pi k} - \sqrt{2\pi k - \frac{\pi}{2} } = 
        \frac{\frac{\pi}{2}}{\sqrt{\vphantom{\frac{\pi}{2}} 2\pi k} + \sqrt{2\pi k - \frac{\pi}{2} }}
        \xrightarrow[k \rightarrow \infty]{} 0 
    .\]
    То есть для любого наперед заданного $\delta > 0$ мы можем найти $k$
    достаточно большое, чтобы разность между $t_1$ и $t_2$ была меньше
    $\delta$, а $\cos t_2^2 - \cos t_1^2 = 1$, что противоречит условию 
    равномерной непрерывности.
\end{Ex}

\begin{Ex}
    А теперь найдем случайную величину $\xi$, такую что $\Expec e^{it\xi} = \cos t$.
    Для этого вспомним, что если $\xi$ "--- дискретная св, то 
    $\Expec g(\xi) = \sum\limits_{k} g(k) \Pro \left( \xi = k \right)  $
    \[
        \cos t = \frac{1}{2} \left( e^{it} + e^{-it} \right)
        \implies \xi = 
        \left\{
            \begin{aligned}
                -1 &, \quad \Pro\left(\xi = -1\right) = \frac{1}{2} \\
                1 &, \quad \Pro \left( \xi = 1 \right) = \frac{1}{2}
            \end{aligned}
        \right. 
    .\] 
\end{Ex} 

\newpage
\end{document}
