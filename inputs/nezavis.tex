\section{Независимость случайных величин}
\begin{Def}
	Случайные величины $\xi, \eta$ называются \underline{независимыми}, если 
	 $$ \forall B_1, B_2 \in \Bor \quad \Pro(\xi \in B_1, \eta \in B_2) = \Pro(\xi \in B_1)\Pro(\eta \in B_2) $$
\end{Def}

\begin{St}
Для независимых случайных величин $\xi, \eta$ выполнено
$$\Expec\xi\eta = \Expec\xi\Expec\eta$$
\end{St}
\begin{Proof}
проведем только для случая дискреьных случайных величин.\\
Пусть 
\[
   	\xi = 
  	\begin{cases}
  		x_1, x_2, \ldots \\
  		p_1, p_2, \ldots
  	\end{cases}
  \]
  То есть $\xi$ приминмает значение $x_i$ с вероятностью $p_i$.
  \[
   	\eta = 
  	\begin{cases}
  		y_1, y_2, \ldots \\
  		q_1, q_2, \ldots
  	\end{cases}
  \]
  $$\Expec\xi\eta=\sum\limits_{i,j} x_iy_j\Pro(\xi=x_i, \eta=y_j) = \sum\limits_{i,j} x_iy_j\Pro(\xi=x_i)\Pro(\eta=y_j) =  \sum\limits_{i} x_i\Pro(\xi=x_i)\sum\limits_j y_j\Pro(\eta=y_j) = \Expec\xi\Expec\eta$$
\end{Proof}

Для независимых случайных величин
$$\Cov(\xi,\eta) = \Expec\xi\eta - \Expec\xi\Expec\eta = 0$$

Обратное, вообще говоря, неверно: пусть мы равновероятно выбираем одну из точек $(-1, 0), (0, 1), (1, 0), (0, -1)$. Каждая координата принимает значения $-1, 0, 1$, но координаты зависимы, так как $\Pro(x=0,  y=0) = 0$ (никогда не выбираем (0,0)), а $\Pro(x=0)\Pro(y=0) = \frac12\frac12 \not=0$. Однако
$$\Expec x = \Expec y = \Expec xy = 0$$ в силу симметрии задачи. Отсюда $\Cov(x,y) = 0$.

Таким образом ковариация показывает зависимость величин, однако не дает представления, насколько они зависимы. Для это вводится понятие коэффициента корреляции ---- нормированная ковариация.

\begin{Def}
\underline{Коэффициент корреляции} --- величина, описываемая формулой
$$\rho(\xi, \eta) = \frac{\Cov(\xi, \eta)}{\sqrt{\Disp\xi\Disp\eta}}$$
\end{Def}

Коэффициент корреляции является псевдоскалярным произведением, то есть выполнены все аксиомы скалярного произведения, кроме половины четвертой. Поэтому для нее выполнено неравенство Коши-Буняковского.

Некоторые свойства коэффициента корреляции:
\begin{enumerate}
	\item $|\rho(\xi, \eta)| \le 1$ --- неравенство Коши-Буняковского
	\item $|\rho(\xi, \eta)| = 1 \Leftrightarrow \exists a, b \colon \quad \Pro(\xi = a\eta + b) = 1$ --- равенство достигается, если случайные величины линейно зависимы
	\item для независимых случайных величин $\rho(\xi, \eta) = 0$
\end{enumerate} 



\newpage

