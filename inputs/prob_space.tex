\documentclass[../TV&MS.tex]{subfiles}
\begin{document}

\section{Вероятностное пространство}

\subsection{Стохастические ситуации}

\qquad Часть информации, изложенной дальше, взято из книги \cite{Korolev} Королева, 
так что любознательный читатель может ознакомиться с первоисточником при особом желании.
Следуя Королеву, будем говорить о \textit{случайности} как о принципиально
неустранимой неопределенности. Чтобы понять, как с ней же работать, необходимо
выделить те случаи, когда мы можем сказать: <<А здесь нам поможет теория вероятностей,>>
~---~и не попасть впросак.

Методы теории вероятности работают в ситуациях, называемых \textit{стохастическими}. 
Для них характерны три свойства:
\begin{enumerate}[label={\bfseries \ding{118}\quad\arabic{enumi}:},leftmargin=*]
	\item \textbf{Непредсказуемость:} исход ситуации нельзя предсказать абсолютно, то
	есть без какой-либо погрешности.
	
	\item \textbf{Воспроизводимость:} у испытателей результатов есть возможность (хотя бы 
	теоретическая) повторить ситуацию сколь угодно много раз при неизменных условиях.
	
	\item \textbf{Устойчивость частот:} при многократном повторении ситуации частота исхода,
	то есть отношение числа опытов, в которых мы получили этот исход, к их общему числу,
	должна колебаться около некоторого значения, приближаясь к нему все ближе и ближе.
	Другими словами, должна иметь частота исходов должна иметь предел при стремлении
	числа опытов к $\infty$.
\end{enumerate}

Для описания стохастических ситуаций ситуаций необходимо определить функцию вероятности. 
Её область определения называется \textit{множеством событий}. В свою очередь событие 
(такое как, например, выпадание чётного числа на кубике) могут являться совокупностью неких 
более простых событий, описывающих стохастическую ситуацию (число, выпавшее на кубике). 
Последнее множество называется \textit{множеством элементарных исходов} и обозначается $\Omega$.  

Есть разные теории описывающие вероятностные пространства, но самой используемой является 
теория, созданная Колмогоровым в начале прошлого века. Перейдем к ее описанию.

\subsection{Измеримое пространство}
	
\qquad Здесь по сути быстрым темпом рассказывается о началах теории меры Лебега,
которая и приведет нас к определению вероятностного пространства. При 
должном желании читатель может ознакомится с целостным и строго математическим
путем повествования меры Лебега, например, здесь~---~\cite{Gusev}.

Вероятностное пространство, которое мы дальше формализуем, должны 
содержать в себе описание трех позиций, важных в эксперименте: 
во-первых, оно должно содержать множество элементарных исходов $\Omega$,
чтобы было предельно понятно, что является исходом; во-вторых,
на этом множестве мы будем рассматривать некоторую структуру подмножеств,
которая будет множеством событий, которые имеют для нас смысл (т.е.
если мы бросаем кубик и нам важна четность выпавшего числа, то не имеет
смысл рассматривать исход <<выпало $1$ очко>> в отдельности от исходов 
<<выпало $3$ очка>> и <<выпало $5$ очков>>); в-третьих, на этом множестве мы 
зададим функцию, которая и будет мерой исхода, то есть некоторой 
характеристикой, описывающей частоту того или иного исхода.


Множество событий, обозначаемое $\Ev$ (используют наравне с этим еще обозначение 
$\mathscr{A}$), должно обладать следующими интуитивными свойствами:
\begin{enumerate}
	\item Отрицание события есть событие (если <<пойдет дождь>> событие, 
	то <<не пойдет дождь>> также событие).
	\item Объединение событий есть событие (<<пойдет дождь>> или <<пойдет снег>>).
	\item Все множество элементарных исходов является событием (<<Что-нибудь да произойдет>>).
\end{enumerate}

Формализуя эти свойства, получаем определение алгебры.
\begin{Def}
	Семейство $\Ev$ подмножеств множества $\Omega$  называется \mdef{алгеброй,} если 
	верны три аксиомы:
\begin{enumerate}[label=(\roman*)]
	\item $\forall A \in \Ev, B\in \Ev \Rightarrow A \cup B \in \Ev$,
	\item $\forall A \in \Ev \Rightarrow \overline{A} \in \Ev$,
	\item $\Omega \in \Ev$.
\end{enumerate}\smallskip
\end{Def}

$\h$ Из аксиом алгебры и формулы $A\cap B = \overline{\overline{A} \cup \overline {B}}$ 
следует, что пересечений событий является событием.

\begin{Ex}
	Наименьшей алгеброй является $\left\{ \Omega, \varnothing \right\}$.
	А наибольшей~---~$\left\{ \Omega, 2^\Omega \right\}$
\end{Ex}

Минус определения, которое мы только что дали, в том, что часто встречаются не 
конечные множества элементарных исходов, поэтому полезно, чтобы множество
событий было замкнуто не только относительно объединения, но и 
относительно счетного объединения.

\begin{Def}
	Семейство $\Ev$ подмножеств множества $\Omega$ называется \mdef{$\sigma$-алгеброй}
	(читается как сигма-алгебра), если 
\begin{enumerate}[label=(\roman*)]
	\item $\forall A_1,\dots,A_n,\ldots\in\Ev\Rightarrow\bigcup\limits_{i=1}^{\infty}A_i\in\Ev$,
	\item $\forall A \in \Ev \Rightarrow \overline{A} \in \Ev$,
	\item $\Omega \in \Ev$.
\end{enumerate}\smallskip
\end{Def}

Не всегда по задаче понятно, какую сигма-алгебру надо выбрать, однако большой
бонус ее аксиом заключается в том, что мы можем пересекать различные сигма-алгебры
и получать снова сигма-алгебру. Из-за этого если нам известно неполное множество
событий $\K$, т.е. $\K\subseteq\Ev$, то мы можем восстановить сигма-алгебру
с точностью до тех событий, которых нет в $\K$. Иными словами, мы можем
найти такую минимальную сигма-алгебру, которая содержит в себе $\K$ как подмножество.

\begin{Def}
	Пусть $\K$~---~класс подмножеств $\Omega$. $\sigma$-алгебра $\sigma(\K)$,
	\mdef{порожденная классом $\K$}~---~наименьшая $\sigma$-алгебра, содержащая $\K$, 
	то есть любая $\sigma$-алгебра, содержащая $\K$, содержит и $\sigma(\K)$.
\end{Def}

\begin{Ex}
	$\sigma$-алгеброй, порожденной $\K = A$, будет являться 
	$\sigma(A) = \left\{ \varnothing, A, \overline{A}, \Omega \right\}$.
\end{Ex}

Сигма-алгебра является более узким понятием, нежели алгебра, то есть любая 
$\sigma$-алгебра является алгеброй, а обратное, вообще говоря, неверно.

\begin{Ex}
	Пусть $\Omega = \Real,\  \Ev$ содержит конечные подмножества $\Omega$ и их дополнения. 
	Для такого множества выполнены все аксиомы алгебры: $\Omega = \overline{\varnothing} \in \Ev$,
	объединение конечных множеств есть конечное множество, объединение конечного множества с 
	дополнением к конечному множеству так же является дополнением к некоторому множеству. 
	То же можно сказать и об объединении двух дополнений. Таким образом, $\Ev$ является алгеброй.
	Все элементы $\Ev$ либо конечны, либо континуальны, поэтому $\Ev$ не содержит $\mathbb{N}$. 
	Но $\mathbb{N} = \bigcup\limits_{i=1}^{\infty}\{i\}$, то есть не выполнено свойство счетной 
	аддитивности из определения $\sigma$-алгебры.
\end{Ex}

	Вот мы и подошли к первому основополагающему определению.

\begin{Def}
	Пара $(\Omega, \Ev)$ называется \mdef{измеримым пространством}, если $\Ev$ является 
	$\sigma$-алгеброй. Если же $\Ev$~---~алгебра, то  $(\Omega, \Ev)$~---~\mdef{измеримое 
	пространство в широком смысле}.
\end{Def}

По сути измеримое пространство нам говорит о том, какие у нас есть элементарные исходы 
и что мы считаем за событие. Однако если бы в известной игре <<Кости>> при броске
кубика каждое число выпадало бы не с равной вероятностью, то это была бы уже совершенно
другая игра, так что имеет смысл определить не только множество событий, но и
вероятность того или иного события. Мы этим и займемся дальше.


\subsection{Вероятность. Вероятностное пространство}
	
\qquad С корабля~---~на бал или, по-современному, не отходя от кассы, 
сразу определим, что такое вероятность.

\begin{Def}
	\mdef{Вероятностью} называется функция $\Pro \colon \Ev\rightarrow \Real$, 
	удовлетворяющая свойствам
\begin{enumerate}
	\item $\forall A \in \Ev \mapsto \Pro (A) \geqslant 0$,
    \item $\forall A_1, \dots, A_n, \ldots \in \Ev, \quad A_i \cap A_j  = \varnothing\  
    (i \not= j)  \Rightarrow \Pro \left(\bigcup\limits_{i=1}^{\infty} A_i\right) = 
    \sum\limits_{i = 1}^{\infty}\Pro(A_i)$,
	\item $\Pro (\Omega) = 1$.
\end{enumerate}
\end{Def}

Второй пункт в определении вероятностной меры нельзя заменить аналогичным с конечными 
объединением и суммой. Однако если добавить к данному требованию так называемое свойство 
непрерывности вероятностной меры, т.е.
$$
	\forall B_1, B_2, \ldots \in \Ev, \quad B_{n+1} \subseteq B_n 
	\Rightarrow \lim_{n \to \infty} \Pro(B_n) = \Pro(B), B = \bigcap_{n=1}^\infty B_n
$$
то они вместе будут эквивалентны 2 из определения вероятности. Покажем это.

\begin{St}
	$\forall A_1, \dots, A_n, \ldots \in \Ev, \quad A_i \cap A_j  = \varnothing\  (i \neq j)
	\Rightarrow \Pro\left(\bigcup\limits_{i=1}^{\infty} A_i\right) = \sum\limits_{i = 1}^{\infty} 
	A_i \Leftrightarrow \left[ \forall A_1, \dots, A_n \in \Ev \quad A_i \cap A_j  = \varnothing\  
	(i \neq j)  \Rightarrow \Pro\left(\bigcup\limits_{i=1}^{n} A_i\right) = \sum\limits_{i = 1}^{n} 
	A_i  \land  \forall B_1, B_2, \ldots \in \Ev,\right.$ \\ $\left. \quad B_{n+1} \subseteq B_n 
	\Rightarrow \lim\limits_{n \to \infty} \Pro(B_n) = \Pro(B), 
	B = \displaystyle\bigcap_{n=1}^\infty B_n\right]$.
\end{St}
\begin{Proof}
\\ ($\Rightarrow$)\\
Обозначим $C_n = B_n \setminus B_{n+1}$. Множества $B, C_1, C_2, \ldots$ не имеют общих точек.\\
$\forall n \quad B_n =  \bigcup\limits_{k=n}^{\infty} C_k \bigcup B$. Тогда $\Pro(B_1) = \Pro(B) 
+ \sum\limits_{k=1}^{\infty} \Pro(C_k)$. Отсюда следует, что ряд в правой части сходится, так как 
имеет конечную сумму. $\Pro(B_n) = \Pro(B) + \sum\limits_{k=n}^{\infty} \Pro(C_k)$. При 
$n \to \infty$ сумма ряда стремится к нулю как остаточный член ряда из предыдущего выражения.
В предельном переходе получаем свойство непрерывности.
\\\\ ($\Leftarrow$)\\
Рассмотрим произвольный набор $A_1, A_2, \ldots \in \Ev \quad A_iA_j = \varnothing$.\\
$\Pro(\bigcup\limits_{i=1}^{\infty} A_i) = \Pro(\bigcup\limits_{i=1}^{n} A_i) + 
\Pro(\bigcup\limits_{i=n + 1}^{\infty} A_i) =\sum\limits_{i = 1}^{n} \Pro(A_i) +  
\Pro(\bigcup\limits_{i=n + 1}^{\infty} A_i) $.\\
Обозначим $B_n = \bigcup\limits_{i=n + 1}^{\infty} A_i,\quad B_{n+1} \subseteq B_n 
\quad \forall n,\quad \bigcap\limits_{n=1}^{\infty} B_n= \varnothing$ \\
$\sum\limits_{i=1}^{\infty} \Pro(A_i) = \lim\limits_{n \to \infty} 
(\Pro(\bigcup\limits_{i=1}^{\infty} A_i) - \Pro(B_n)) = \Pro(\bigcup\limits_{i=1}^{\infty} A_i) 
- \lim\limits_{n \to \infty} \Pro(B_n) = \Pro(\bigcup\limits_{i=1}^{\infty} A_i)$
\end{Proof}

\begin{Def}
	\mdef{Вероятностным пространством} $(\Omega, \Ev, \Pro)$ называется измеримое 
	пространство $(\Omega, \Ev)$, снабженное вероятностью $\Pro$.
\end{Def}
	
Как видите: и здесь нет ничего сложного. Идет по сути просто жонглирование математическими
понятиями и задание одних определений.
	
\begin{Wtf}
	Кому вообще нужна $\sigma$-алгебра событий и зачем весь этот огород, если можно 
	рассматривать множество всех подмножеств множества событий $\Omega$? Когда-то давно кто-то 
	доказал, что в случае очень большого множества элементарных исходов, например, 
	континуального, множество $2^{\Omega}$ будет иметь такую крокодильски большую мощность, 
	что вся теория сломается. Таким образом, алгебры нужны для того, чтобы вероятность имела 
	хорошую область определения.
\end{Wtf}

	Перечислим свойства вероятности. Доказательства их можно найти в любом из
	классических учебников по теории вероятностей или можно их придумать самому:
	большинство из них тривиальны.

\textbf{Свойства вероятности}:
\begin{enumerate}
	\item $\Pro (\varnothing) = 0$, 
	\item $\Pro (\overline{A}) = 1 - \Pro (A)$,
	\item $A \subseteq B \Rightarrow \Pro (A) \le \Pro (B)$,
	\item $\Pro (A) \le 1$,
	\item $\Pro (A \cup B) = \Pro (A) + \Pro (B) - \Pro (AB)$, 
	\item $\Pro (A \cup B) \leqslant \Pro (A) + \Pro (B)$,
	\item $\Pro \left(\bigcup\limits_{i=1}^{n} A_i\right) = 
	\sum\limits_{k=1}^{n} \sum\limits_{i_1<\dots <i_k} (-1)^{k+1} 
	\Pro(A_{i_1}A_{i_2}\ldots A_{i_k})$,
	\item $\Pro \left(\bigcap\limits_{i=1}^{n} A_i\right) \geqslant 1 - 
	\sum\limits_{i=1}^{n} \Pro (\overline{A_i})$~---~неравенство Бонферрони.
\end{enumerate}

Если порождать сигма-алгебру алгеброй, то вероятность будет определена
однозначно вероятностью на данной алгебре. Это и есть содержание
следующей теоремы.

\begin{Th}[Каратеодори]
Пусть $(\Omega, \Ev)$ --- измеримое пространство в широком смысле, 
а некоторая функция $\Pro$ обладает свойствами вероятностной меры. 
Тогда на измеримом пространстве $(\Omega, \sigma(\Ev))$
$$
	\exists !\  \Pro' \colon \forall A \in \Ev \mapsto \Pro(A) = \Pro'(A).
$$
\end{Th}
\begin{Proof}
	См. \cite{Gusev}.
\end{Proof}

\begin{Why}
	Зачем это нужно? Теорема Каратеодори говорит о том, что любую вероятностную меру, 
	заданную на алгебре, можно однозначно продолжить на $\sigma$-алгебру, то есть расширить 
	область ее определения. При этом значения функции на алгебре не изменятся. Теорема будет 
	использоваться при определении интеграла Лебега.
\end{Why}
\newpage
\end{document}