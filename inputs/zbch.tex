\documentclass[../TV&MS.tex]{subfiles}
\begin{document}
    
\section{Закон больших чисел}
В этой части рассмотрим закон больших чисел. Причем сначала скажем, как он формулируется, а потом подоказываем его для различных исходных данных.

\begin{Def}
    Пусть $\{\xi_n\}$ "--- последовательность независимых одинаково распределенных случайных величин, а $S_n = \sum\limits_{k=1}^{n} \xi_n$.
    Говорят, что $\{\xi_n\}$ удовлетворяет \uline{закону больших чисел (ЗБЧ)}, если $\dfrac{S_n}{n} \xrightarrow[]{\Pro} \Expec \xi_i.$ 
\end{Def} 

\begin{Ex}
    Пусть $\xi_k$ независимы и равномерно распределены на отрезке $[-1, 1]$. Тогда матожидание каждого слагаемого равно $0$. Тогда
    \[
        \Pro \left( \left| \dfrac{S_n}{n} \right| \geqslant \varepsilon \right) \leqslant \dfrac{\Disp S_n}{\varepsilon^2} = \dfrac{\Disp \xi_1}{n \varepsilon^2} \rightarrow 0 
    .\]
    То есть $\dfrac{S_n}{n}$ сходится по вероятности к своему матожиданию, значит для данной последовательности слуайных величин выполнен ЗБЧ.
\end{Ex} 

\begin{Th}[ЗБЧ в форме Хинчина]
    Если $\{ \xi_k \}$ "--- независимые одинаково распределенные случайные величины с конечным матождиданием, то для них выполнен ЗБЧ. 
\end{Th} 

\begin{Proof}
    Первое желание, которое может возникнуть при виде данной теоремы "--- записать неравенство Чебышева, но для неравентва Чебышева нужно не только конечное матожидание, но и конечный второй момент, поэтому так не прокатит.
    Тогда мы воспользуемся тем, что если последовательность случайных величин сходится к числу, то сходимость по вероятности эквивалентна слабой сходимости, и в качестве функции в определении слабой сходимости положим характеристическую функцию.
     \[
         f(t) := \Expec e^{it\xi_k} \quad \forall k=\overline{1 \ldots n}  
     .\]
     Здесь нам не важно, для какой случайной величины считать харфункцию или матожидание, так как они все одинаково распределены.
     \[
         f_n(t) := \Expec e^{it\frac{S_n}{n}} = \Expec e^{it \frac{\xi_1 + \ldots + \xi_n}{n}} = \prod\limits_{k=1}^{n} \Expec e^{\frac{it\xi_k}{n}} = 
         f^{n} \left( \frac{t}{n} \right)
    .\]
    Далее вспомним важной свойство харфункции: $a := f'(0) = \Expec \xi_k$. Тогда: 
    \[
        f^{n} \left( \dfrac{t}{n} \right) = \left( 1 + \dfrac{ita}{n} + \overline{o} \left( \dfrac{1}{n} \right)  \right)^{n} \xrightarrow[n \rightarrow \infty ]{} e^{ita} 
    .\]
    Заметим, что $e^{ita}$ непрерывна на всей действительной прямой и обращается в 1 при $t = 0$. Значит по теореме о непрерывности $e^{ita}$ представляет собой харфункцию некоторой случайной величины, к которой слабо сходится $\dfrac{S_n}{n}$. Действительно, эта случайная величина $\xi \xequiv{\text{п.н.}} a$.
    Таким образом, имеем, что $\dfrac{s_n}{n} \xrightarrow[]{w} \Expec \xi_k$,
    но если предельная случайная величина вырождена, то слабая сходимость эквивалентна сходимости по вероятности, значит $\dfrac{s_n}{n} \xrightarrow[]{\Pro} \Expec \xi_k$.
\end{Proof} 

Лулзов ради закинем еще парочку ЗБЧ.

\begin{Th}[ЗБЧ в форме Чебышева]
    Пусть $\{ \xi_k \}$ "--- последовательность некоррелированных случайных величин с равномерно ограниченной дисперсией.
    То есть $\sup\limits_{k \in \mathbb{N}} \Disp \xi_k \leqslant C < \infty$.
    Тогда для данной последовательности случайых величин выполнен ЗБЧ.
\end{Th}

\begin{Proof}
    Последовательно воспользуемся неравенством Чебышева, некоррелированостью случайных величин (а значит дисперсия суммы равна сумме дисперсий) и равномерной ограниченностью дисперсий.
    \[
    \Pro \left( \left| \dfrac{S_n}{n} - \Expec \xi_k \right| \geqslant \varepsilon \right) = 
    \Pro \left( \left| \dfrac{S_n - \Expec S_n}{n} \right| \geqslant \varepsilon \right) \leqslant
    \dfrac{\Disp S_n}{(n\varepsilon)^2} =
    \dfrac{1}{n^2\varepsilon^2} \sum\limits_{k=1}^{n} \Disp \xi_k \leqslant
    \dfrac{C}{n\varepsilon^2} \xrightarrow[n \rightarrow \infty]{} 0
    .\] 
\end{Proof} 

\begin{Note}
    Условия теоремы можно ослабить: вместо некоррелированности потребовать, чтобы сумма ковариаций обращалась в 0, а вместо равномерной ограниченности константой можно потребовать, чтобы дисперсии росли строго медленнее, чем линейная функция.
\end{Note} 

\begin{Th}[Усиленный ЗБЧ (УЗБЧ) в форме Колмогорова]
    Если $\{ \xi_k \}$ "--- независимые одинаково распределенные случайные величины с конечным матождиданием, то $\dfrac{S_n}{n} \xrightarrow[]{\text{п.н.}}  \Expec \xi_k$. 
\end{Th}

\begin{Note}
    Условия те же, что и на ЗБЧ в форме Хинчина, но сходимость уже не по вероятности, а почти наверное.
\end{Note} 

\begin{Proof}
    Все верно. Отвечаю.
\end{Proof} 

\begin{Wtf}
    Собственно, а к чему будут сходиться средние арифметические, если нет матожидания?
\end{Wtf} 

\begin{Ex}
    Распределим $\{ \xi_k \}$ по Коши с параметрами $C(0, \gamma)$ и посмотрим, что будет происходить со средними арифмитическими.
    \[
        p_{\xi_k} (x) = \dfrac{1}{\pi \gamma \left( 1 + \left( \dfrac{x}{\gamma} \right) ^2 \right) }, \quad
        f(t) :=  \Expec e^{it\xi_k} = e^{-\gamma \left| t \right| }
    .\]

    \begin{multline*}
        f_n(t) := \Expec S_n = 
        \Expec e^{it \sum\limits_{k=1}^{n} \xi_k} = 
        \prod\limits_{k=1}^{n} \Expec e^{it\xi_k} = 
        f^{n} \left( t \right) =
        e^{-\gamma n \left| t \right| } \implies \\
        p_{S_n} \left( t \right) = \dfrac{1}{\pi n \gamma \left( 1 + \left( \dfrac{x}{n \gamma} \right)^2 \right) }
    .\end{multline*}
    Теперь предположим, что последовательность $\{ S_n \} $ сходится по вероятности к чему-то, и придем к противоречию.
    Пусть $\Pro \left( \left| \dfrac{S_n}{n} - a \right| \geqslant \varepsilon \right) \rightarrow 0$. Тогда:
    \begin{multline*}
        \Pro \left( \left| \dfrac{S_n}{n} - a \right| \geqslant \varepsilon \right) = \int\limits_{n(a+\varepsilon)}^{+\infty} p(x)dx + \int\limits_{-\infty}^{n(a - \varepsilon)} p(x)dx = 
        \dfrac{1}{\pi} \left. \arctg \left( \dfrac{x}{\gamma n} \right) \right|_{n(a+\varepsilon)}^{+\infty} + 
        \dfrac{1}{\pi} \left. \arctg \left( \dfrac{x}{\gamma n} \right) \right|_{-\infty}^{n(a - \varepsilon)} = \\
        = \dfrac{1}{\pi} \left( \arctg \left( \dfrac{a - \varepsilon}{\gamma} \right) - \arctg \left( \dfrac{a + \varepsilon}{\gamma} \right)  \right) 
        \xcancel{ \xrightarrow[n \rightarrow \infty]{}} \ 0
    .\end{multline*}
    Таким образом, последовательность средних арифметических случайных величин, распределенных по Коши, не сходится по вероятности вообще ни к чему.
\end{Ex} 

\newpage


\end{document}
