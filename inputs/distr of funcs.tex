\section{Распределение функций от случайных величин}
Рассмотрим отображение $f\colon \Real^n \to \Real^n$ с якобианом $J_f = \frac{D(f_1(x), \ldots, f_n(x))}{D(x_1, \ldots, x_n)}$.
Если он отличен от 0, то существует обратная функция, и выполнено соотношение
$$J_fJ_{f^{-1}} = 1$$

\begin{Th}
Пусть  $f\colon \Real^n \to \Real^n$ --- достаточно гладкая функция с ненулевым якобианом. $\xi = (\xi_1, \ldots, \xi_n) \sim p_\xi(x)$. Рассмотрим случайную величину $\eta = f(\xi)$. Тогда
$$p_\eta(x) = p_\xi(f^{-1}(x))|J_{f^{-1}}(x)|$$
\end{Th}
\begin{Proof} \\
Вспомним формулу замены переменных в интеграле:
$$\int\limits_B \phi(x)dx = \int\limits_{f^{-1}(B)} \phi(f(y))|J_f(y)|dy$$
Тогда $\forall B \in \Bor_n$
$$\int\limits_B p_\xi(f^{-1}(x))|J_{f^{-1}(x)}|dx = \int\limits_{f^{-1}(B)} p_\xi(f(f^{-1}(B)))|J_{f^{-1}(x)}||J_{f(x)}|dx =$$
$$ \int\limits_{f^{-1}(B)}p_\xi(x)dx = \Pro(\xi \in f^{-1}(B)) =\Pro(f(\xi) \in B) = \Pro(\eta \in B)$$
\end{Proof}

Теперь найдем формулу для плотности суммы случайных величин.
Пусть $\xi = (\xi_1, \xi_2) \sim p_\xi(x)$
Рассмотрим $f\colon (x_1, x_2) \mapsto (x_1 + x_2, x_2)$. Тогда $$f^{-1} \colon (x_1, x_2) \mapsto (x_1 - x_2, x_2), \quad J_{f^{-1}}=1$$
Обозначим $\eta = f(\xi)$. Тогда по только что доказанной теореме 
$p_\eta(x_1, x_2) = p_{\xi}(x_1 - x_2, x_2)$
Тогда по свойству 4 функции распределения векторной случайной величины
$$p_{\xi_1 + \xi_2}(x) = \int\limits_{-\infty}^{+\infty}p_\xi(x-x_2, x_2)dx_2 = \text{\{независимость $\xi_1, \xi_2$\}} = \int\limits_{-\infty}^{+\infty}p_{\xi_1}(x-x_2)p_{\xi_2}(x_2)dx_2$$
В силу симметрии 
$$p_{\xi_1 + \xi_2}(x) = \int\limits_{-\infty}^{+\infty}p_{\xi_1}(x_1)p_{\xi_2}(x - x_2)dx_1$$

Эти формулы называются \underline{формулами свертки}. Существует аналогичная формула для функция распределения, но мы ее не доказываем, потому что это какой-то функан:
$$F_{\xi_1 + \xi_2}(x) = \int\limits_{-\infty}^{+\infty}F_{\xi_1}(x-y)dF_{\xi_2}(y) = \int\limits_{-\infty}^{+\infty}F_{\xi_2}(x-y)dF_{\xi_1}(y)$$

\newpage

