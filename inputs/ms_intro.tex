\documentclass[../TV&MS.tex]{subfiles}
\begin{document}

\section{Введение}

В прошлой части мы упарывались с теорией вероятности, а теперь будем флексить
над математической статистикой. Отличительной особенностью теории вероятности
было то, что нам дана была вероятностная мера и мы должны были что-то там с
ней сделать. Например, найти вероятность какой-то сложного события или 
матожидание какой-нибудь случайной величины. А в матстате вероятностная мера
не задана в условии, и надо как-то извратиться и найти ее или хотя бы оценить.
Просто так с потолка что-либо оценить довольно трудно, поэтому оценивают
обычно на основе нескольких реализаций случайных событий. Теперь любимые
определения:

\begin{Def}
    Множество случайных величин $\Smpl = \left( X_1, \ldots X_n \right)$
    называется \uline{выборкой}, если все слечайные величины независимы в 
    совокупности.
\end{Def} 

\begin{Def}
    Выборка называется \uline{однородной}, если случайные величины одинаково
    распределены.
\end{Def} 
В дальнейшем под выборкой мы будем подразумевать однородную выборку, т.\,е.
случайные величины будут всегда (если не оговорено отдельно) норсв.
Теперь, раз уж у нас есть много случаных величин, то можно рассматривать
выборку как векторную случайную величину, определенную на измеримом
пространстве $ \left( \Real_n, \Bor_n \right) $. Но, как мы помним,
вероятностной меры мы не знаем, поэтому на данном вероятностном пространстве
зададим сразу семейство вероятностных мер $\mathcal{P}$ и получится такой
крокодил, как

\begin{Def}
    \uline{Статистическая структура} "--- $ \left( \Real_n, \Bor_n,
    \mathcal{P}\right) $.
\end{Def} 
Есть 3 больших класса задач, которые решает матстат:

\begin{enumerate}
    \item \textit{Точечное оценивание}. В задачах точеченого оценивания мы
        считаем, что наше вероятностное семейство параметризовано некоторым
        параметром $\theta$, изменяющемся на множестве $\Theta$:
        $\mathcal{P} = \Set{\Prob_{\theta}}{\theta \in \Theta}$.
        В этом классе задач от нас хотят, чтобы мы сказали, что 
        $\theta \approx \theta_0 \in \Theta$, где примерное равенство обычно
        понимают в смысле математического ожидания. Иногда так же полезно
        найти разброс (дисперсию) нашей оценки. Например, мы знаем, что
        выборка сгенерирована из нормального распределение с дисперсией $1$ и
        неизвестным матожиданием, которое надо найти.
    
    \item \textit{Интервальное оценивание}. Здесь у нас тоже вероятность
        параметризована, мно мы уже не просто тыкаем в какое-то конкретное
        значение $\theta_0$, а говорим, что $\theta_1 < \theta < \theta_2$ с
        такой-то вероятностью.

    \item \textit{Проверка гипотез}. Гипотеза "--- это какое-то наше
        предположение относительно вероятностных мер. Обозначается $H$ 
        (возможно, с индексом) и формализуется каждый раз по-разному.
        Напирмер, гипотеза о том, что выборка из равномерного распределения
        или о том, что выборка из гамма-распределения с параметром формы, не
        превосходящем $13$.
\end{enumerate} 

\end{document}
