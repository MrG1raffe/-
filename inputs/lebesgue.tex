\documentclass[../TV&MS.tex]{subfiles}
\begin{document}
\section{Понятие меры и интеграла Лебега}

\qquad Мера Лебега вводится потихонечку: сначала для полуинтервала, потом для борелевских множеств, а затем и для всей числовой прямой. Основная идея --- обобщение
понятия длины на страшные, ненормальные множества.

Вспомним сначала аксиомы меры $\mu$ из функана (хех): это функция на множестве $B$ (в нашем случае $[0, 1)$), $\mu \colon B \to \Real$ со свойствами
\begin{enumerate}
	\item $\forall A \in B \quad \mu(A) \ge 0$
	\item $\forall A_1, A_2, \ldots \in B \colon A_iA_j = \varnothing \quad (i \not= j) \Rightarrow \mu(\bigcup\limits_{i=1}^\infty A_i) = \sum\limits_{i=1}^\infty \mu(A_i)$
\end{enumerate} 

Введем сначала меру на борелевских множествах полуинтервала $[0, 1)$, то есть на $\Bor_{[0, 1)}$. Итак, 
\begin{itemize}
	\item На полуинтервалах $(a, b)$ введем меру как $\mu_{[0, 1)}((a, b)) = b - a$
	\item Тогда мера одной точки равна нулю, и мы можем не обращать внимание на концы множеств
	\item Любое конечное объединение, пересечение, отрицание таких интервалов есть конечное объединение интервалов (и, возможно, конечное число точек-концов)
	На них введем меру как сумму мер этих интервалов.
	\item Теперь осталось определить меру на бесконечных объединениях/пересечениях. Для этого воспользуемся теоремой Каратеодори, согласно которой можно продолжить
	меру на $\sigma$-алгебре.
\end{itemize}

Аналогичным образом определим меру $\mu_{[k, k+1)}$ на всех полуинтервалах вида $[k, k+1) \quad (k \in \mathbb{Z})$.
Осталось доопределить меру на всей $\Bor$. $\forall A \in \Bor \quad$
 $$\mu(A)=\sum\limits_{i=-\infty}^{\infty}\mu_{[i, i+1)}(A \bigcap [i, i+1))$$
 
 Теперь введем интеграл \underline{интеграл Лебега} (интеграл по мере Лебега). В отличие от интеграла Римана, где происходит разбиение области определения и выбирается
 значение функции из образа элемента разбиения, в интеграле Лебега разбивается область значений (то есть $Oy$), и некоторое значение из элемента разбиения умножается на
 меру прообраза этого элемента, затем все благополучно складывается. Введем теперь это формально.
 
 Будем рассматривать функции $f \colon \Real \to \Real$ такие, что $\forall c \in \Real \quad \Set{x}{f(x) < c} \in \Bor$, то есть прообразы полупрямых являются
  измеримыми множествами. А как было показано выше, отсюда следует, что прообразы всех борелевских множеств являются измеримыми.
  
  Введем сначала понятие \underline{индикатора}. Индикатор события $A$ - это случайная величина, принимающая значение 1, если событие произошло, и 0 в противном случае.
  Таким образом
  \[
  	\Ind(x) = 
  	\begin{cases}
  		1, x \in A \\
  		0, x \not\in A
  	\end{cases}
  \]
  
 Определим интеграл Лебега на полуинтервале $[0, 1)$
\begin{itemize}
 	\item Пусть функция имеет вид $f(x) = \sum\limits_{i=1}^ky_i \Ind(A_i) \quad (i \not= j \Rightarrow A_iA_j = \varnothing, \ \bigcup_{i=1}^k A_i= [0,1))$. Функции такого вида 		называются \underline{финитными}.  В этом случае
 		$$\int\limits_0^1f(x)\mu(dx) = \sum\limits_{i=1}^k y_i \mu(A_i) $$
	То есть берется мера кусочка, на котором функция принимает значение $y_i$, и умножается на это значение. Получается нечто, напоминающее площадь под графиком.
	\item Пусть теперь $f(x) \ge 0$ на $[0, 1)$. Будем приближать функцию финитной. Возьмем отрезок области значений $[0, n], n \in \mathbb{N}$. Разобьем этот отрезок на
	$n2^n$ кусочков $[\frac{k-1}{2^n}, \frac{k}{2^n}], k = 1, 2, \dots, n2^n$. На каждом кусочке скажем, что значение функции равно $\frac{k-1}{2^n}$. Осталось приблизить 
	$[n, +\infty)$. Будем считать это одной частью, на которой функция принимает значение $n$. Осталось устремить n к бесконечности: тогда, так как размер каждого кусочка первой части равен $\frac{1}{2^n}$, их длина станет бесконечно малой, а та часть, которую мы <<обрубили>> сверху (то есть $[n, +\infty)$), уйдет в бесконечность. Итак,
	$$ \int\limits_0^1 f(x) \mu(dx) = \lim_{n \to \infty}  \biggl[  \sum\limits_{k=1}^{n2^n} \frac{k-1}{2^n} \mu\biggl(\Set{x}{\frac{k-1}{2^n} \le f(x) < \frac{k}{2^n}}\biggl) 
	+ n \mu(\Set{x}{f(x) \ge n})\biggl]$$
	\item Остался случай произвольной функции $f(x)$. Разобьем ее на две: одна функция совпадает с $f(x)$ там, где та положительна, и равняется 0 в остальных случаях,
	другая равна $|f(x)|$ там, где $f(x)$ отрицательна, и 0 иначе. 
	$$ f^+(x) = \max\{0, f(x)\},\quad f^-(x) = -\min\{0, f(x)\}$$
	Для каждой из этих функций интеграл определен по предыдущему пункту. Пользуясь тем, что $f(x) = f^+(x) - f^-(x)$, определим интеграл так:
	$$  \int\limits_0^1 f(x) \mu(dx) =  \int\limits_0^1 f^+(x) \mu(dx) - \int\limits_0^1 f^-(x) \mu(dx) $$
	В случае, если оба интеграла в правой части расходятся, интеграл от $f(x)$ не определен. 
	Так как $|f(x)| = f^+(x) + f^-(x)$, сходимость интеграла Лебега от модуля функции (абсолютная сходимость) эквивалента сходимости интеграла от самой функции
	(то есть обычной сходимости). Это следует из того, что интеграл сходится только в случае конечности обоих интегралов правой части (иначе он не определен), откуда 
	следует конечность их суммы и разности. Таким образом, для интеграла Лебега не существует условно сходящихся функций.
\end{itemize}

	Аналогично вводим интеграл на каждом полуинтервале $[i, i+1), \quad i \in \mathbb{Z}$. Тогда на всей прямой интеграл по множеству $A \subset \Real$ будет определяться так:
	$$ \int\limits_A f(x) \mu(dx) = \sum\limits_{i=-\infty}^{+\infty} \int\limits_{A \bigcap [i, i+1)} f(x) \mu(dx) $$
\begin{Ex}
С помощью интеграла Лебега можно считать интегралы от функций, об интегрировании которых раньше было страшно даже подумать. Например, от функции Дирихле:
  \[
  	D_{[0,1)}(x) = 
  	\begin{cases}
  		1, x \in [0,1) \setminus \mathbb{Q} \\
  		0, x \in \mathbb{Q}
  	\end{cases}
  \]
  Данная функция является финитной, а именно $D_{[0,1)}(x) = \Ind([0,1) \setminus \mathbb{Q})$. Поэтому по первому пунккту
  $$ \int\limits_0^1 D_{[0,1)}(x) \mu(dx) = 1 $$
\end{Ex}

  В дальнейшем $\mu(dx)$ будет опускаться обозначаться просто как $dx$ или $dy$ чтобы подчеркнуть, что считается именно интеграл Лебега.
  
\newpage
\end{document}
