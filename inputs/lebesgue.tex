\documentclass[../TV&MS.tex]{subfiles}
\begin{document}
\section{Понятие меры и интеграла Лебега}

\subsection{Мера Лебега}

\qquad Не пугайтесь. Сейчас мы введем меру Лебега, но не так быстро, как 
вводили вероятностное пространство. Представьте, что задание меры Лебега~---~это
как надуть несколько воздушных шариков внутри друг друга. Самый маленький 
шарик~---~это полуинтервалы. Средний~---~борелевские множества (о них будет
дальше). Большой~---~измеримые множества на прямой. На каждом из этих множеств
вводится своя мера, а потом продолжается на большие множества.

Дадим определение семейства, которое нас будет сильно интересовать дальше.

\begin{Def}
\mdef{Борелевской $\sigma$-алгеброй} называется минимальная $\sigma$-алгебра, 
содержащая все открытые подмножества топологического пространства. Элементы 
борелевской $\sigma$-алгебры называются \mdef{борелевскими множествами}.
\end{Def}

\begin{Wtf}
Мы будем рассматривать только топологическое пространство $\Real$, так что это стремное 
словосочетание можно прямо сейчас забыть и понимать открытое множество как открытое множество 
из матана (все точки внутренние).
\end{Wtf}

\begin{Ex}
Покажем, что все <<хорошие>> множества являются Борелевскими.
\begin{enumerate}
\item Все открытые интревалы входят по определению.
\item Отрезок вида $[a, b]$ входит как $\overline{(-\infty, a) \cup (b, +\infty)}$.
\item Точка ходит как вырожденный отрезок $[a, a]$.
\item Счетное объединение таких множеств входит по определению.
\end{enumerate}
\end{Ex}

\begin{Def}
Теперь вспомним сначала аксиомы \mdef{счетно-аддитивной меры $\mu$} 
из функана (злодейский смех): это функция на множестве 
$B$, $\mu \colon B \to \Real$, со свойствами
\begin{enumerate}
	\item $\forall A \in B \mapsto \mu(A) \ge 0$ (неотрицательность),
	\item $\forall A_1, A_2, \ldots \in B \colon A_i\cap A_j = \varnothing 
	\quad (i \not= j) \Rightarrow \mu\left(\bigcup\limits_{i=1}^\infty A_i\right) =
	\sum\limits_{i=1}^\infty \mu(A_i)$ ($\sigma$-аддитивность).
\end{enumerate} 
\end{Def}

Начнем использовать это определение в самом простом случае. Введем сначала меру на 
борелевских множествах полуинтервала $[0, 1)$, то есть на $\Bor_{[0, 1)}$. Итак, 
\begin{itemize}
	\item На интервалах $(a, b)$ введем меру как $\mu_{[0, 1)}((a, b)) = b - a$,
	то есть примерно если бы ваш сосед пришел бы с рулеткой к прямой и, померив, 
	грубым голос добавил: <<Здесь на вскидку $b-a$ получается.>>

	\item Мера одной точки равна нулю, и мы можем не обращать внимание на 
	концы множеств. Это так пусть будет по определению.
	
	\item На любых множествах, которые можно представить как конечное
	объединение интервалов и какого-то числа точек, мы зададим меру как 
	сумму мер каждого из составляющих. Это как если бы вам надо было 
	добраться по проспекту от точки $A$ до точки $B$: прошли до остановки 
	метров $100$, сели в автобус, проехали пару километров, вышли, пересели
	в другой автобус, проехали на нем, вышли, и прошли до места назначения
	метров $200$. Тогда множество точек проспекта (прямой), где вы шли или
	ждали автобус, будет иметь меру (длину) равную $300$ метрам:
	$$300 = 100 + 0 + 200.$$

	\item Теперь осталось определить меру на бесконечных объединениях/пересечениях. 
	Для этого воспользуемся теоремой Каратеодори \mref{karateodori}, согласно 
	которой можно продолжить меру на $\sigma$-алгебру.
\end{itemize}

Аналогичным образом определим меру $\mu_{[k, k+1)}$ на всех полуинтервалах 
с целыми концами $[k, k+1), (k \in \mathbb{Z})$. Осталось доопределить меру на всей 
$\Bor$.
$$
	\forall A \in \Bor \quad\mu(A)=\sum\limits_{i=-\infty}^{\infty}\mu_{[i, i+1)}
	(A \cap [i, i+1)).
$$
 
\subsection{Интеграл Лебега} 
 
Теперь введем \emph{интеграл Лебега} (интеграл по мере Лебега). В отличие 
от интеграла Римана, где происходит разбиение области определения и выбирается
значение функции из образа элемента разбиения, в интеграле Лебега разбивается 
область значений (то есть $Oy$), и некоторое значение из элемента разбиения умножается 
на меру прообраза этого элемента, затем все благополучно складывается. Введем теперь 
это формально.
 
Будем рассматривать функции $f \colon \Real \to \Real$ такие, что $\forall c\in\Real$ 
множество $A_c\Set{x}{f(x) < c}$ борелевское, то есть прообразы полупрямых являются
борелевскими множествами.
  
Введем сначала понятие \emph{индикатора}. Индикатор события $A$ --- это случайная 
величина, принимающая значение $1$, если событие произошло, и $0$ в противном случае.
Таким образом
\[
  	\Ind(x) = 
  	\begin{cases}
  		1, x \in A, \\
  		0, x \not\in A.
  	\end{cases}
\]
  
Определим интеграл Лебега на полуинтервале $[0, 1)$
\begin{itemize}
 	\item Пусть функция имеет вид $f(x) = \sum\limits_{i=1}^ky_i \Ind(A_i)$,
 	где $A_i$ не пересекаются и покрывают весь полуинтервал $[0,1)$. 
 	Функции такого вида называются \emph{финитными}. В этом случае
 	$$
 		\int\limits_0^1f(x)\mu(dx) = \sum\limits_{i=1}^k y_i \mu(A_i). 
 	$$
	
	То есть берется мера кусочка, на котором функция принимает значение $y_i$, 
	и умножается на это значение. Получается нечто, напоминающее площадь под графиком.

	\item Пусть теперь $f(x) \ge 0$ на $[0, 1)$. Будем приближать функцию финитной. 
	Возьмем отрезок области значений $[0, n], n \in \mathbb{N}$. Разобьем этот отрезок 
	на $n2^n$ кусочков $[\frac{k-1}{2^n}, \frac{k}{2^n}), k = 1, 2, \dots, n2^n$. 
	На каждом кусочке скажем, что значение функции равно $\frac{k-1}{2^n}$. 
	Осталось приблизить $[n, +\infty)$. Будем считать это одной частью, на которой 
	функция принимает значение $n$. Устремим $n$ к бесконечности: тогда, так 
	как размер каждого кусочка первой части равен $\frac{1}{2^n}$, их длина станет 
	бесконечно малой, а та часть, которую мы <<обрубили>> сверху (то есть $[n, +\infty)$), 
	уйдет в бесконечность. Итак,
	$$ 
		\int\limits_0^1 f(x) \mu(dx) = \lim_{n \to \infty}  
		\left[\sum\limits_{k=1}^{n2^n} \frac{k-1}{2^n} 
		\mu\left(\Set{x}{\frac{k-1}{2^n} \le f(x) < \frac{k}{2^n}}\right) 
		+ n \mu(\Set{x}{f(x) \ge n})\right].
	$$
	
	\item Остался случай произвольной функции $f(x)$. Разобьем ее на две: одна 
	функция совпадает с $f(x)$ там, где та положительна, и равняется $0$ в остальных 
	случаях, другая равна $|f(x)|$ там, где $f(x)$ отрицательна, и $0$ иначе. 
	$$ 
		f^+(x) = \max\{0, f(x)\},\quad f^-(x) = -\min\{0, f(x)\}.
	$$

	Для каждой из этих функций интеграл определен по предыдущему пункту. 
	Пользуясь тем, что $f(x) = f^+(x) - f^-(x)$, определим интеграл так:
	$$  
		\int\limits_0^1 f(x) \mu(dx) =  \int\limits_0^1 f^+(x) \mu(dx) - 
		\int\limits_0^1 f^-(x) \mu(dx). 
	$$

	В случае, если оба интеграла в правой части расходятся, интеграл от $f(x)$ не 
	определен. Так как $|f(x)| = f^+(x) + f^-(x)$, сходимость интеграла Лебега от 
	модуля функции (абсолютная сходимость) эквивалента сходимости интеграла от самой 
	функции (то есть обычной сходимости). Это следует из того, что интеграл сходится 
	только в случае конечности обоих интегралов правой части (иначе он не определен), 
	откуда следует конечность их суммы и разности. Таким образом, для интеграла 
	Лебега не существует условно сходящихся функций.
\end{itemize}

Аналогично вводим интеграл на каждом полуинтервале $[i, i+1), i \in \mathbb{Z}$. 
Тогда на всей прямой интеграл по множеству $A \subset \Real$ будет определяться так:
$$ 
	\int\limits_A f(x) \mu(dx) = \sum\limits_{i=-\infty}^{+\infty} 
	\int\limits_{A \bigcap [i, i+1)} f(x) \mu(dx). 
$$
	
\begin{Ex}
	С помощью интеграла Лебега можно считать интегралы от функций, об интегрировании 
	которых раньше было страшно даже подумать. Например, от функции Дирихле:
  	\[
  		D_{[0,1)}(x) = 
  		\begin{cases}
  			1, x \in [0,1) \setminus \mathbb{Q}, \\
  			0, x \in \mathbb{Q}.
  		\end{cases}
  	\]
  	
	Данная функция является финитной, а именно $D_{[0,1)}(x) = \Ind([0,1) 
	\setminus \mathbb{Q})$. Поэтому по первому пункту
 	$$ 
 		\int\limits_0^1 D_{[0,1)}(x) \mu(dx) = 1. 
 	$$
\end{Ex}

В дальнейшем $\mu(dx)$ будет опускаться обозначаться просто как $dx$ или 
$dy$ чтобы подчеркнуть, что считается именно интеграл Лебега.

Все. На этом введение в теорию меры заканчивается. Перейдем к более 
интересным разделам: связанным уже с вероятностями.

\newpage
\end{document}
