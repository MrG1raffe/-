\documentclass[../TV&MS.tex]{subfiles}
\begin{document}
    
\section{Случайные величины}

\qquad Случайные события "--- это хорошо, но с события типа <<на монетке выпал герб>> плохо формализуемы, а мы хотим формальности и математичности. Поэтому вместо всяких событий хочется работать с числами. Вот этим и займемся. При рассмотрении случайных событий мы ввели вероятностное пространство, которе выглядит так:
$$(\Omega, \Ev, \Pro),$$
где $\Omega$ "--- множество элементарных событий, $\Ev$ "--- $\sigma$-алгебра подмножеств множества элементарных событий, а $\Pro$ "--- вероятность. Мы же будем рассамтривать теперь тройку
$$(\Real, \Bor, \Pro),$$
где $\Real$ "--- действительная прямая, $\Bor$ "--- борелевская $\sigma$-алгебра, а $\Pro$ "--- вероятность. Поясним.

\begin{Def}
\underline{Борелевской $\sigma$-алгеброй} называется минимальная $\sigma$-алгебра, содержащая все открытые подмножества топологического пространства. Элементы борелевской $\sigma$-алгебры называются \underline{борелевскими множествами}.
\end{Def}
\begin{Wtf}
Мы будем рассматривать только топологическое пространство $\Real$, так что это стремное словосочетание можно прямо 
сейчас забыть и понимать открытое множество как открытое множество из матана (все точки внутренние).
\end{Wtf}\\
\begin{Ex}
Покажем, что все <<хорошие>> множества являются Борелевскими.
\begin{enumerate}
\item Все открытые интревалы входят по определению.
\item Отрезок вида $[a, b]$ входит как $\overline{(-\infty, a) \bigcup (b, +\infty)}$.
\item Точка ходит как вырожденный отрезок $[a, a]$.
\item Счетное объединение таких множеств входит по поределению.
\end{enumerate}
\end{Ex}

Теперь формально введем понятие случайной величины (может использоваться сокращение с.в.).

\begin{Def}
Пусть $(\Omega, \Ev, \Pro)$ "--- вероятностное пространство.\\ Тогда \underline{случайной величиной $\xi$} называется функция $\xi : \Omega \to \Real$, измеримая относительно $\Ev$ и $\Bor$. По-другому, $\xi$ "--- случайная величина, если
$$\forall B \in \Bor \quad \xi^{-1}(B) = \lbrace \omega : \xi(\omega) \in B \rbrace \in \Ev$$.
\end{Def}
\begin{Wtf}
Таким финтом ушами мы, по сути, сопоставили каждому событию какое-то <<хорошее>> множество на числовой прямой, и можем рассматривать не вероятности событий, а вероятности попадания в эти <<хорошие>> подмножества числовой прямой.
\end{Wtf}

Введем еще несколько бесполезных определений, которые в дальнейшем использоваться не будут, но знать их не вредно.

\begin{Def}
С каждой случайной величиной свяжем два вероятностных пространства: первое --- $(\Omega, \Ev_\xi, \Pro)$ --- вероятностное пространство, \underline{порожденное $\xi$}. Здесь 
$\Ev_\xi$ - наименьшая $\sigma$-алгебра, для которой выполняется свойство измеримости. Второе --- $(\Real, \Bor, \Pro_\xi)$, где $\Pro_\xi(B) = \Pro(\xi^{-1}(B)) \quad \forall B \in \Bor$ и называется \underline{распределением вероятностей $\xi$}.
\end{Def}

Идем дальше в~сторону упрощения работы со случайностями. Вместо того чтобы рассматривать произвольные борелевские множества, мы будем рассамтривать только множества вида $(-\infty, x)$. Действительно, интервал $(a, b)$ получается из~полупрямых так: $(a, b) = (-\infty, b) \setminus (-\infty, a]$  Таким образом, мы можем рассматривать случайные величины только на таких множествах. Здесь имеется в виду, что для удовлетворения определению случайной величины достаточно измеримости только на
 полупрямых, что следует из следующих свойств полного прообраза: прообраз объединения есть объединение прообразов, прообраз пересечения есть пересечение прообразов,
 прообраз отрицания есть отрицание прообраза. Выше показано, что из полупрямых с помощью этих операций можно получить интервалы, а из интервалов и все $\Bor$.

Теперь несколько полезных утверждений. Пусть $\xi$ --- случайная величина. Тогда $-\xi$ также случайная величина, так как её прообраз от любой полупрямой является
прообразом $\xi$ от симметричной полупрямой, то есть лежит в $\Ev$. $\xi + c$ также будет случайной величиной, поскольку ее прообразом для любой полупрямой будет
прообраз $\xi$ для полупрямой, сдвинутой на $c$, то есть будет лежать в $\Ev$.

\begin{St}
Пусть $\xi, \eta$ --- случайные величины. Тогда множество $\Set{\omega}{\xi(\omega) < \eta(\omega)}$ является событием.
\end{St}
\begin{Proof}
$\Set{\omega}{\xi(\omega) < И\eta(\omega)} = \bigcup\limits_{r \in \mathbb{Q}}\Set{\omega}{\xi(\omega) < r, \eta(\omega) > r}$. 
Заметим, что $\Set{\omega}{\xi(\omega) < r)}$ является событием. Аналогично для $\eta$. Выражение, написанное выше, является счетным объединением пересечений двух событий, то есть событием.
\end{Proof}

Похожими махинациями, а также с использованием этого утверждения, доказывается, что $\xi^2, \xi + \eta, \xi\eta$ являются случайными величинами.
Более того, если $\xi_1, \ldots, \xi_n$ --- с.в., а функция $\phi(x_1, \ldots, x_n)$ является непрерывной на множестве их значений, то $\phi(\xi_1, \ldots, \xi_n)$ будет случайной 
величиной. Это не доказывалось.

\begin{Def}
Рассмотрим вероятностное пространство $(\Omega, \Ev, \Pro)$ и определенную на нем случайную величину $\xi$. Тогда её \underline{функцией распределения $F_\xi(x)$}
 называется функция $F_\xi : \Real \to \Real$
$$F_\xi(x) = \Pro(\omega : \xi(\omega) < x) = \Pro(\xi < x)$$.
\end{Def}

Запись $\Pro(\xi < x)$ является в некотором смысле жаргонной, так как аргументов вероятности должно быть событие из $\Ev$. Но $\xi < x$ мы в дальнейшем будем 
отождествлять с оъединением элементарных событий, образ которых меньше $x$. Из определения случайной величины получаем, что это объединение является событием,
поэтому применение к нему функции вероятности корректно.

Функция распределения (сокращение ф.р.) является очень полезной штукой, поскольку имеет достаточно простой вид и несет в себе всю информацию о распределении, то есть однозначно 
определяет $\Pro_\xi$.

Рассмотрим основные свойства функции распределения:
\begin{enumerate}
	\item $F_\xi(x) \in [0, 1]$
	\item $\lim\limits_{x \to -\infty} F_\xi(x) = 0$
	\item $\lim\limits_{x \to +\infty} F_\xi(x) = 1$
	\item $F_\xi(x)$ монотонно не убывает.
	\item $F_\xi(x)$ непрерывна слева.
\end{enumerate}

Вероятность попадания с.в. в полуинтервал $\Pro_\xi[a,b) = F_\xi(b) - F_\xi(a)$. При стремлении $b \to a$ получим $\Pro(\xi = a) = F_\xi(a+) - F_\xi(a)$, то есть 
вероятность попадания в точку равна скачку функции распределения в этой точке.

\begin{Def}
Точка $x_0$ называется \underline{точкой роста} $F_\xi(x)$, если $\forall \epsilon > 0 \quad \Pro(x_0 - \epsilon \le \xi < x_0 + \epsilon) > 0$
\end{Def}
\begin{Ex}
Это очень полезный пример, который будет использоваться в матстате и который очень любят спрашивать. Пусть $\xi$ --- случайная величина. $\eta = F_\xi(\xi)$. Чему равна
$F_\eta(x)$? По определению $F_\eta(x) = \Pro(\eta < x) = \Pro(F_\xi(\xi) < x)=\Pro(\xi < F_\xi^{-1}(x)) = F_\xi(F_\xi^{-1}(x))=x$. Вообще, тут было бы неплохо 
сказать, что $F_\xi$ непрерывна и строго монотонна, чтобы со спокойной совестью использовать обратную функцию. Таким образом $\eta$ имеет равномерное распределение.
\end{Ex}

\newpage



\end{document}
