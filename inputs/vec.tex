\section{Векторные случайные величины}
Векторную случайную величину можно определить двумя способами:
\begin{enumerate}
	\item Сказать, что $\xi = (\xi_1, \ldots, \xi_n)$ является векторной случайной величиной, если ее координаты являются случайными величинами.
	\item Дать определение аналогично определению одномерной случайной величины, то есть рассмотреть $(\Real^n, \Bor_n)$, где $\Bor_n$ --- $n$-мерная борелевская
	$\sigma$-алгебра, то есть минимальная $\sigma$-алгебра, содержащая все параллелепипеды (аналогично интервалам в одномерном случае). Тогда $\xi$ --- случайная величина, если $$\forall B \in \Bor_n \colon \xi^{-1}(B) \in \Ev$$. 
\end{enumerate}

\begin{Def}
\underline{Функция распределения} $$F_\xi(x_1, \ldots, x_n) = \Pro(\xi_1 < x_1, \ldots, \xi_n < x_n)$$
\end{Def}

Свойства функции распределения:
\begin{enumerate}
	\item Пусть $x, y \in \Real^n$ такие, что $x_i < y_i \quad i = 1, \ldots, n$. Тогда $F_\xi(x) \le F_\xi(y)$.
	\item $0 \le F_\xi(x) \le 1$
	\item $\lim\limits_{x_i \to -\infty} F_\xi(x) = 0$. То есть при стремлении одной координаты к $-\infty$ функция распределения стремится к 0, поскольку вероятность для соответсвующей координаты стремится к 0.
	\item Если же какую-то из координат устремить к бесконечности, то она не будет учитываться в вероятности, поэтому 
	$\lim\limits_{x_i \to +\infty} F_\xi(x) = F_{(\xi_1, \ldots, \xi_{i-1}, \xi_{i+1}, \ldots, \xi_n)}(x_1, \ldots, x_{i-1}, x_{i+1}, \ldots, x_n)$.
	\item Непрерывна слева по каждому аргументу.
\end{enumerate}

\begin{Def}
\underline{Абсолютно непрерывная} случайная величина $\xi$ --- такая случайная величина, что
$$\forall B \in \Bor_n \quad \Pro(\xi \in \Bor_n) = \int\limits_B p_\xi(x_1, \ldots, x_n)dx_1\ldots dx_n$$ 
\end{Def}

$\xi_1, \ldots, \xi_n$ независимы в совокупности $\Leftrightarrow$ функцию распределения векторной случайной величины $\xi = (\xi_1, \ldots, \xi_n)$ можно представить произведением функций распределений координат, то есть 
$$F_\xi(x) = F_{\xi_1}(x_1)\ldots F_{\xi_n}(x_n)$$
В случае абсолютно непрерынвой векторной случайной величины это эквивалентно аналогичному выражению для плотностей:
$$p_\xi(x) = p_{\xi_1}(x_1)\ldots p_{\xi_n}(x_n)$$

\newpage

