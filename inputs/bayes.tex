\documentclass[../TV&MS.tex]{subfiles}
\begin{document}
    
\section{Формула полной вероятности. Формула Байеса}

\qquad Чуть выше мы рассмотрели условную вероятность, то есть поговорили о том, что
некоторые события могут пересекаться и когда в таких случаях можно сказать, что они
независимы. Часто встречается следующая группа событий.	
	
\begin{Def}
$B_1, \ldots, B_n$ образуют \mdef{полную группу}, если выполнены следующие условия:
	\begin{enumerate}
		\item $\Pro(B_i) > 0 \quad \forall i = 1, \ldots, n$,
		\item $B_iB_j = \varnothing \quad (i \not= j)$,
		\item $\bigcup\limits_{i=1}^nB_i = \Omega$.
	\end{enumerate}
\end{Def}

\begin{Th}
Пусть $B_1, \ldots, B_n$ образуют полную группу. Вероятность события $A \in \Ev$ можно 
вычислить по \emph{формуле полной вероятности}:
$$\Pro(A) = \sum\limits_{i=1}^{n} \Pro(A|B_i)\Pro(B_i)$$
\end{Th}

\begin{Proof}
$$A=\bigcup\limits_{i=1}^{n}AB_i, \quad AB_i \cap AB_j = \varnothing \quad (i \not= j),$$
$$\Pro(A) = \sum\limits_{i=1}^n \Pro(AB_i) = \sum\limits_{i=1}^{n} \Pro(A|B_i)\Pro(B_i).$$
Последний переход следует из  закона умножения вероятностей.
\end{Proof}

Первое требование определения полной группы необходимо для возможности определить 
условную вероятность, второе позволяет разбить множество $A$ на непересекающиеся части. 
Третье требование, вообще говоря, можно ослабить, потребовав, чтобы 
$A \subseteq \bigcup\limits_{i=1}^nB_i$. Доказательство при этом не изменится. \\

\begin{Ex}
Проиллюстрировать формулу полной вероятности можно обычным экзаменом: $A$ --- \{студент сдал экзамен\}, 
$B_i$ --- \{студент попал к преподавателю $i$\}. Как и в любой другой лотерее, можно оценить вероятность 
попадания к преподавателю $i$, то есть $\Pro(B_i)$, а трезво оценивая свои силы можно прикинуть 
и вероятность сдать тому или иному преподавателю $\Pro(A|B_i)$. Зная все вышеперечисленное, 
несложно по формуле вычислить вероятность успешной сдачи.
\end{Ex}

Формула полной вероятности используется для вычисления априорной вероятности, т.е. вероятности 
события, которое  еще не произошло. Пусть теперь $\Pro(A) > 0$. Тогда 
$\Pro(B_i|A)=\Pro(AB_i)/\Pro(A)$. Используя формулу полной вероятности, 
получаем \emph{формулу Байеса}:
$$\Pro(B_i|A)=\frac{\Pro(A|B_i)\Pro(B_i)}{\sum\limits_{j=1}^n\Pro(A|B_j)\Pro(B_j)}$$

Формула Байеса используется для вычисления апостериорной вероятности. То есть уже известно, что произошло некоторое событие $A$, и нужно вычислить вероятность
того, что произошло некоторое $B_i$. В примере с экзаменом, например, может быть известно, что студент не сдал экзамен, и хочется вычислить вероятность того, что он
сдавал преподавателю <<$P$>>.

\newpage


\end{document}
