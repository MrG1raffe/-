\documentclass[../TV&MS.tex]{subfiles}
\begin{document}
    
\section{Характеристики случайных величин}

\underline{Математическое ожидание} (обозначается $\Expec$) обобщает понятие среднего арфиметического для произвольной случайной величины и показывает, какие значения в среднем принимает случайная величина. Оно, как и интеграл Лебега, вводится в несколько этапов. В этом определении вероятность $\Pro$ играет роль
Лебеговой меры $\mu$.

\begin{itemize}
 	\item Если $\xi(\omega) = \sum\limits_{i=1}^kx_i \Ind(A_i) \quad (i \not= j \Rightarrow A_iA_j = \varnothing, \ \bigcup_{i=1}^k = \Omega)$. 
 	$$\Expec \xi = \int\limits_{\Omega} \xi(\omega) \Pro(d\omega) \equiv \sum\limits_{i=1}^k x_i \Pro(\Set{\omega}{\xi(\omega)=x_i}) =  \sum\limits_{i=1}^k x_i \Pro(A_i)$$
	\item $\xi(\omega) \ge 0$. В этом случае аналогично интегралу Лебега
	$$ \Expec \xi = \int\limits_{\Omega} \xi d\Pro = \lim_{n \to \infty}  \biggl[  \sum\limits_{k=1}^{n2^n} \frac{k-1}{2^n} \Pro\biggl(\Set{\omega}{\frac{k-1}{2^n} \le \xi(\omega) < \frac{k}{2^n}}\biggl) 
	+ n \Pro(\Set{\omega}{\xi(\omega) \ge n})\biggl]$$
	\item Для произвольной $\xi(\omega)$ вводятся
	$$ \xi^+(\omega) = \max\{0, \xi(\omega)\},\quad \xi^-(\omega) = -\min\{0, \xi(\omega)\}$$
	$$ \Expec \xi = \Expec \xi^+ - \Expec \xi^-$$
\end{itemize}

Вспомним, что любая случайная величина $\xi$ индуцирует вероятностное пространство $(\Real, \Bor_\xi, \Pro_\xi)$. Тогда, поскольку $\Pro_\xi(dx)$ есть вероятность попасть
в $dx$, выразим ее через функцию распределения $\Pro_\xi(dx) = F_\xi(x + dx) - F_\xi(x) = dF_\xi(x)$ Тогда перепишем матожидание в более привычной форме
$$\Expec\xi = \int\limits_{\Omega} \xi(\omega) \Pro(d\omega) = \int\limits_{-\infty}^{+\infty} x \Pro_\xi(dx) = \int\limits_{-\infty}^{+\infty} x dF_\xi(x)$$

Перечислим некоторые свойства математического ожидания:
\begin{enumerate}
	\item $\Expec(\xi + a) = \Expec\xi + a \quad \forall a \in \Real$
	\item $\Expec(a\xi) = a\Expec\xi \quad \forall a \in \Real$
	\item $\Expec(\xi + \eta) = \Expec\xi + \Expec\eta$ (здесь подразумевается, что существуют два из трех математических ожидания, из чего следует существование третьего)
\end{enumerate}
\begin{Ex}
Рассмотрим дискретную случайную величину $\xi$, которая принимает значения $n$ с вероятностью $\frac{c}{n^2} \ (c = \frac{6}{\pi^2})$. По определению
$$ \Expec\xi = \sum\limits_{n=1}^{\infty} n \frac{c}{n^2} $$
Данный ряд, очевидно, расходится. Сиутацию не спасет даже рассмотрение случайной величины $\eta$, принимающей значения $\pm n$ с вероятностью $\frac{c}{2n^2}$, 
которая имеет среднее значение 0, 
поскольку $\Expec\eta = \sum\limits_{n=1}^{\infty} \frac{c}{2n} - \sum\limits_{n=1}^{\infty} \frac{c}{2n}$, что не определено, поскольку интегралы Лебега от $\eta^+$ и $\eta^-$ расходятся.\\
\end{Ex}
\begin{Ex}
Другим примером является распределение Коши с плотностью $p_\xi(x) = \frac{1}{\pi(1+x^2)}$. График этой функции симметричен относительно 0 и похож на горку, из чего
методом пристального взгляда можно сделать вывод, что средним значением должно быть 0. Однако  $
\int\limits_{-\infty}^{+\infty} x dF_\xi(x) =  \int\limits_{-\infty}^{+\infty} x p_\xi(x)dx$ расходится, поэтому математического ожидания не существует.
\end{Ex}

\begin{Def}
\underline{Моментом порядка $k$} случайной величины $\xi$ называется $\Expec\xi^k$ \\
\underline{Абсолютным моментом порядка $k$} случайной величины $\xi$ называется $\Expec|\xi|^k$ \\
\underline{Центральным моментом порядка $k$} случайной величины $\xi$ называется $\Expec(\xi - \Expec\xi)^k$ \\
\end{Def}

\begin{Def}
\underline{Квантилью} случайной величины $\xi$ порядка $q$ называется величина $l_\xi(q)$:
\[
   	l_\xi(q) \colon 
  	\begin{cases}
  		\Pro(\xi \le l_\xi(q)) \ge q \\
  		\Pro(\xi \ge l_\xi(q) \ge 1-q
  	\end{cases}
  \]
  
  В случае $q=\frac12$ квантиль называется \underline{медианой} и обозначается $\Med\xi$.
  
  Если $q=\frac14$, то $l_\xi$ называется \underline{квартилью}, если $q=\frac1{10}$, \underline{децилью}, а если $q=\frac{1}{100}$ --- \underline{перцентилью}.
\end{Def}

Жизненный смысл медианы заключается в том, что она является точкой, вероятность попасть левее которой равна пероятности попасть правее нее. Аналогично можно 
сказать про квантиль любого порядка. Квантиль определена не единственным образом: пусть $\xi$ принимает значения $\{0, 1\}$ с вероятностями $\frac12$. Тогда медианой
$\xi$ может быть любая точка из отрезка $[0,1]$.

\begin{Def}
\underline{Интерквантильный размах} --- величина $R_\xi = l_\xi(\frac34) - l_\xi(\frac14)$ --- длина отрезка, вероятность попасть в который равна $\frac12$. 
\end{Def}

С матожиданием и медианой связана задача о <<деловых людях>>. (здесь будет ссылка) 

\begin{Def}
\underline{Мода} случайной величины $\xi$ --- это наиболее вероятное значение случайной величины. Оюозначается $\Mod\xi$.
\end{Def}

При наблюдении случайной величины важно знать не только её среднее значение (матожидание), но и то, как сильно она от него отклоняется (например, измерение линейнкой в среднем дает правильный результат, однаком необходимо знать погрешность измерения). В связи с этим вводится понятие дисперсии.

\begin{Def}
Пусть для случайной величины $\xi$ существуют конечный $\Expec\xi$ и $\Expec\xi^2$.  \underline{Дисперсией} назывется величина, равная 
$$ \Disp\xi = \Expec(\xi - \Expec\xi)^2$$
\end{Def}

Эту формулу можно привести к более простому для вычисления виду:
$$\Disp\xi = \Expec(\xi^2 - 2\xi\Expec\xi + (\Expec\xi)^2) = \Expec\xi^2 - 2\Expec\xi\Expec\xi - (\Expec\xi)^2 = \Expec\xi^2 - (\Expec\xi)^2$$


Перечислим некоторые свойства дисперсии. $\xi,\eta$ - случайные величины, $c \in \Real$.

\begin{enumerate}
	\item $\Disp\xi \ge 0$ как матожидание от неотрицательной функции
	\item $\Disp c\xi = c^2 \Disp\xi$ --- следует из определения и линейности матожидания.
	\item $\Disp(\xi+c) = \Expec(\xi + c - \Expec(\xi + c))^2 = \Expec(\xi - \Expec\xi)^2 = \Disp\xi$
	\item $\Pro(\xi = c) = 1 \Leftrightarrow \Disp\xi = 0$ --- отклонение равно нулю для константы
	\item $\Disp(\xi + \eta) = \Expec(\xi+\eta)^2 - (\Expec(\xi+\eta))^2 = \Expec(\xi^2+2\xi\eta + \eta^2) - (\Expec\xi)^2- 2\Expec\xi\Expec\eta - (\Expec\eta)^2 = \Disp\xi + \Disp\eta + 2(\Expec\xi\eta - \Expec\xi\Expec\eta)$
\end{enumerate}

\begin{Def}
\underline{Ковариация} двух случайных величин --- это величина, равная
$$ \Cov(\xi,\eta) = \Expec(\xi - \Expec\xi)\Expec(\eta - \Expec\eta)$$
\end{Def}

Ковариация положительна, если случайные величины одновременно отклоняются в одну сторону и отрицательная, если в разные. Формулу ковариации так же можно упростить, раскрыв скобки в определении:
$$\Cov(\xi,\eta) = \Expec\xi\eta - \Expec\xi\Expec\eta$$

Таким образом пятое свойство дисперсии можно переписать так:
\begin{itemize}
	\item[5.] $\Disp(\xi \pm \eta) = \Disp\xi + \Disp\eta \pm 2\Cov(\xi,\eta)$
\end{itemize}

\begin{Def}
\underline{Среднеквадратическое отклонение} --- величина $\sigma = \sqrt{\Disp\xi}$.
\end{Def}



\newpage


\end{document}
